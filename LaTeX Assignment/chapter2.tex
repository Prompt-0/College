% This is your second chapter.
\chapter{Linear Algebra} % Chapter 2 title from the reference book

\section*{Abstract}
%
% This abstract describes the "Linear Algebra" chapter (Chapter 2)
% of the "Deep Learning" book.
%
Linear algebra is a crucial branch of mathematics for understanding
and working with machine learning algorithms, especially deep learning.
This chapter provides the essential mathematical foundation for
the rest of the book.
We will see that the core data objects
we manipulate are not single numbers, but tensors (multidimensional arrays).
These objects include scalars (0D tensors), vectors (1D tensors),
and matrices (2D tensors).
A batch of images, for example, can be represented as a 4D tensor.
The fundamental operations within a neural network, such as the
transformation in a fully-connected layer, are expressed
using matrix multiplication.
The parameters, or 'weights', of a network are themselves stored as matrices.
Understanding how these objects interact is therefore essential.
We will review key operations like the transpose and the dot product.
Furthermore, we will cover foundational concepts like eigendecomposition
and Singular Value Decomposition (SVD).
These concepts are not just theoretical;
they are the basis
for powerful algorithms like Principal Components Analysis (PCA).
PCA itself is.
a simple example of a representation learning algorithm,
a core theme of this book.
This chapter builds the vocabulary we need to describe deep models.
Without a solid grasp of these linear algebra concepts, it is
nearly impossible to understand how deep learning models are
structured, how they process data, or how they are trained.
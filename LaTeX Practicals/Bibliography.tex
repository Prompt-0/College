% This document demonstrates Task 6:
% Creating a manual bibliography using the 'thebibliography'
% environment and citing references in the text.

\documentclass[11pt, a4paper]{article}

% --- UNIVERSAL PREAMBLE BLOCK ---
\usepackage[a4paper, top=2.5cm, bottom=2.5cm, left=2cm, right=2cm]{geometry}
\usepackage{fontspec}
\usepackage[english, bidi=basic, provide=*]{babel}
\babelprovide[import, onchar=ids fonts]{english}
\babelfont{rm}{Noto Sans}
% --- END UNIVERSAL PREAMBLE BLOCK ---

\begin{document}

\section{Introduction}

LaTeX provides built-in tools for managing references. We can cite sources directly in the text. For example, Knuth~\cite{knuth1984} wrote the book on \TeX.

Other authors have also contributed significantly. Lamport~\cite{lamport1994} created LaTeX itself. Goossens et al.~\cite{goossens1994} wrote an essential guide.

It is possible to cite multiple sources at once~\cite{knuth1984, lamport1994, tolsby2000}. A final reference might be on a related topic~\cite{kopka2003}.

\section{Conclusion}

The \texttt{thebibliography} environment automatically generates the "Bibliography" title and formats the list, linking the \texttt{\string\cite} commands to the \texttt{\string\bibitem} entries.

% This environment creates the bibliography
% The '9' is a placeholder that tells LaTeX the widest label will be
% a single digit, for alignment. Use '99' for double digits.
\begin{thebibliography}{9}

\bibitem{knuth1984}
D. E. Knuth. (1984).
\textit{The \TeX book}.
Addison-Wesley.

\bibitem{lamport1994}
L. Lamport. (1994).
\textit{\LaTeX: A Document Preparation System}.
Addison-Wesley.

\bibitem{goossens1994}
M. Goossens, F. Mittelbach, \& A. Samarin. (1994).
\textit{The \LaTeX\ Companion}.
Addison-Wesley.

\bibitem{kopka2003}
H. Kopka \& P. W. Daly. (2003).
\textit{Guide to \LaTeX}.
Addison-Wesley.

\bibitem{tolsby2000}
M. Tolsby. (2000).
\textit{My Fictional Book on Typesetting}.
Publisher of Interest.

\end{thebibliography}

\end{document}
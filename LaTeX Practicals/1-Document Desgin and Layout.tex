% This document demonstrates Task 1:
% Document design, title, author, date, address, page dimensions,
% margins, line spacing, footnotes, and orientation.

% We choose 'article' class. '11pt' for font size, 'a4paper' for paper size.
\documentclass[11pt, a4paper]{article}

% --- UNIVERSAL PREAMBLE BLOCK ---
% Use 'geometry' to set margins and orientation (landscape)
\usepackage[landscape, top=2cm, bottom=2cm, left=2.5cm, right=2.5cm]{geometry}
\usepackage{fontspec} % Allows use of modern fonts

% Setup Babel for language support (English)
\usepackage[english, bidi=basic, provide=*]{babel}
\babelprovide[import, onchar=ids fonts]{english}

% Set default font to Noto Sans (a clean, modern font)
\babelfont{rm}{Noto Sans}
% --- END UNIVERSAL PREAMBLE BLOCK ---

% Use 'setspace' to control line spacing
\usepackage{setspace}

% --- Title and Author Information ---
\title{Demonstration of Document Design}
\author{Your Name}
% We can add an address or institution inside the \date command
% Replaced \par with \\ which is the correct way to break lines here.
\date{\today \\ \medskip \textit{123 University Ave, LaTeX City}}

\begin{document}

% This command generates the title block
\maketitle

% Apply one-and-a-half line spacing to the document
\onehalfspacing

\section{Introduction}

This document demonstrates several key layout features. The page orientation is set to 'landscape', and the margins have been customized using the \texttt{geometry} package.

We are also using \texttt{setspace} to set the line spacing to one-and-a-half. This can be changed to \texttt{\doublespacing} or \texttt{\singlespacing} as needed.

This is a sentence that includes a footnote.\footnote{This is the text of the footnote. It appears at the bottom of the page.}

\section{Conclusion}

By adjusting the preamble, you have fine-grained control over the entire look and feel of your document before you even write the first word.

\end{document}
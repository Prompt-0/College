% This document demonstrates Task 5:
% Typesetting fine-tuned math, aligning and numbering equations,
% and using symbols from 'amsmath'.

\documentclass[11pt, a4paper]{article}

% --- UNIVERSAL PREAMBLE BLOCK ---
\usepackage[a4paper, top=2.5cm, bottom=2.5cm, left=2cm, right=2cm]{geometry}
\usepackage{fontspec}
\usepackage[english, bidi=basic, provide=*]{babel}
\babelprovide[import, onchar=ids fonts]{english}
\babelfont{rm}{Noto Sans}
% --- END UNIVERSAL PREAMBLE BLOCK ---

% 'amsmath' is essential for advanced math
\usepackage{amsmath}
% 'amssymb' provides additional math symbols
\usepackage{amssymb}

\begin{document}

\section{Fine-Tuning Math Expressions}

Inline math, like $f(x) = x^2 + \sqrt{y}$, is part of a text line. Display math is centered on its own line:
$$
\sum_{i=0}^{n} i^2 = \frac{n(n+1)(2n+1)}{6}
$$
We can add fine-tuned spacing, like a thin space (\texttt{\,}) in an integral: $\int f(x) \, dx$ versus $\int f(x) dx$.

\section{Aligning and Numbering Equations}

The \texttt{align} environment is used to align multiple equations at the \texttt{\&} symbol. It also numbers them automatically.
\begin{align}
  E &= mc^2 \label{eq:einstein} \\
  a^2 + b^2 &= c^2 \label{eq:pythagoras} \\
  f(x) &= x^3 + 2x^2 - 5x + 1 \notag \\
       &\quad + \text{a long line broken with alignment}
\end{align}

We can reference these equations, like Equation~\eqref{eq:einstein} or Equation~\eqref{eq:pythagoras}.
The \texttt{\notag} command suppresses the number for that line.
The \texttt{\text{...}} command lets us put normal text inside math.

\section{Using Math Symbols}

The \texttt{amsmath} and \texttt{amssymb} packages provide many symbols.
\begin{itemize}
    \item From \texttt{amsmath}: $\boxed{y = mx + b}$ (boxed equation), $\implies$ (implies arrow).
    \item From \texttt{amssymb}: $\mathbb{R}$ (set of real numbers), $\mathcal{F}$ (calligraphic F), $\approx$ (approximately equal), $\geq$ (greater or equal).
\end{itemize}

\end{document}